% Copyright 2013 Christophe-Marie Duquesne <chmd@chmd.fr>
% Copyright 2014 Mark Szepieniec <http://github.com/mszep>
% 
% ConText style for making a resume with pandoc. Inspired by moderncv.
% 
% This CSS document is delivered to you under the CC BY-SA 3.0 License.
% https://creativecommons.org/licenses/by-sa/3.0/deed.en_US

\startmode[*mkii]
  \enableregime[utf-8]  
  \setupcolors[state=start]
\stopmode

\setupcolor[hex]
\definecolor[titlegrey][h=757575]
\definecolor[sectioncolor][h=397249]
\definecolor[rulecolor][h=9cb770]

% Enable hyperlinks
\setupinteraction[state=start, color=sectioncolor]

\setuppapersize [A4][A4]
\setuplayout    [width=middle, height=middle,
                 backspace=20mm, cutspace=0mm,
                 topspace=10mm, bottomspace=20mm,
                 header=0mm, footer=0mm]

%\setuppagenumbering[location={footer,center}]

\setupbodyfont[11pt, helvetica]

\setupwhitespace[medium]

\setupblackrules[width=31mm, color=rulecolor]

\setuphead[chapter]      [style=\tfd]
\setuphead[section]      [style=\tfd\bf, color=titlegrey, align=middle]
\setuphead[subsection]   [style=\tfb\bf, color=sectioncolor, align=right,
                          before={\leavevmode\blackrule\hspace}]
\setuphead[subsubsection][style=\bf]

\setuphead[chapter, section, subsection, subsubsection][number=no]

%\setupdescriptions[width=10mm]

\definedescription
  [description]
  [headstyle=bold, style=normal,
   location=hanging, width=18mm, distance=14mm, margin=0cm]

\setupitemize[autointro, packed]    % prevent orphan list intro
\setupitemize[indentnext=no]

\setupfloat[figure][default={here,nonumber}]
\setupfloat[table][default={here,nonumber}]

\setuptables[textwidth=max, HL=none]

\setupthinrules[width=15em] % width of horizontal rules

\setupdelimitedtext
  [blockquote]
  [before={\setupalign[middle]},
   indentnext=no,
  ]


\starttext

\section[kevin-caye]{Kevin Caye}

\thinrule

\startblockquote
Computer Science - Statistics - Machine Learning - Optimization - Data
Visualization
\stopblockquote

\thinrule

\subsection[compétences]{Compétences}

\startdescription{Informatique}
  {\bf Programmation :} R, python, C/C++, SQL.

  {\bf OS :} Linux, OS X.

  {\bf Outils :} Emacs, Org mode, Git, RStudio, Jupyter.
\stopdescription

\startdescription{Maths}
  Optimisation, Statistiques, Modélisation Probabiliste, Apprentissage
  Automatique.
\stopdescription

\startdescription{Langues}
  {\bf Anglais :} bon niveau (lu, écrit, parlé).

  {\bf Espagnol :} niveau moyen (lu, écrit, parlé).
\stopdescription

\subsection[formation]{Formation}

\startdescription{2014-oct. 2017}
  {\bf Thèse en mathématiques appliquées}; TIMC-IMAG, La Tronche (38)

  {\em Méthodes d'apprentissage statistique pour les tests d'association
  écologique}
\stopdescription

\startdescription{2011-2014}
  {\bf Diplôme d'ingénieur en informatique et mathématiques appliquées};
  Grenoble-INP, ENSIMAG, Saint-Martin D'Heres (38)

  {\em Filière : MMIS (Modélisation Mathématique, Image et Simulation)}
\stopdescription

\startdescription{2008-2011}
  {\bf Classe préparatoire MPSI/MP*}; lycée Albert schweitzer, Le Raincy
  (93)
\stopdescription

\subsection[expériences-professionnelles]{Expériences professionnelles}

\startdescription{2014-oct. 2017}
  {\bf Thèse en mathématiques appliquées.} Les manières de représenter
  la santé et le vivant ont considérablement changé durant ces dernières
  années. Ces nouvelles représentations sont basées sur la collecte de
  données massives, et plus particulièrement de données
  épidémiologiques, démographiques et biologiques issues de
  plates-formes à haut débit. L'objectif de cette thèse est de
  développer des méthodes permettant de détecter des signatures
  d'association entre variables dans de telles données. Nous nous
  intéressons notamment aux modèles basées sur des problèmes
  d'optimisation faisant intervenir des factorisations matricielles.
\stopdescription

\startdescription{2014-2016}
  {\bf Enseignement.} Deux ans de monitorat à l'Université Grenoble
  Alpes en L1 :

  \startitemize[packed]
  \item
    TP de découverte des mathématiques appliquées
  \item
    TD de soutien de mathématiques
  \item
    Cours/TP de modélisation de la dynamique en biologie
  \stopitemize

  Encadrement de projet à l'ensimag.
\stopdescription

\startdescription{Février-Août 2014}
  {\bf Stage recherche et developement}; Dassault Systèmes,
  Vélizy-Villacoublay (78). Dans le département {\em Science and
  Corporate Research}, le stage consistait à faire un état de l'art des
  méthodes de simulation 3D basées sur des exemples. Par la suite un
  prototype a été implémenté à l'aide d'OpenGL et C++.
\stopdescription

\startdescription{Juillet-Août 2013}
  {\bf Stage ingénieur R&D }; INRIA, Montbonnot (38). Dans l'équipe
  MAVERICK, le stage consistait à implementer des techniques de rendu 3D
  à l'aide de la bibliothèque Gigavoxel (C++, GLSL, CUDA).
\stopdescription

\subsection[publications]{Publications}

{\bf Fast Inference of Individual Admixture Coefficients Using
Geographic Data}\crlf
Kevin Caye, Flora Jay, Olivier Michel, Olivier François\crlf
{\em bioRxiv}; doi:
\useURL[url1][http://dx.doi.org/10.1101/080291][][10.1101/080291]\from[url1]

{\bf Identifying outlier loci in admixed and in continuous populations
using ancestral population differentiation statistics}\crlf
Helena Martins, Kevin Caye, Keurcien Luu, Michael GB Blum, Olivier
François\crlf
{\em bioRxiv}; doi:
\useURL[url2][http://dx.doi.org/10.1101/054585][][10.1101/054585]\from[url2]

{\bf Controlling false discoveries in genome scans for selection}\crlf
Olivier François, Helena Martins, Kevin Caye, Sean D. Schoville\crlf
{\em Molecular ecology}, 2016, vol.~25, no 2, p.~454-469; doi:
\useURL[url3][http://dx.doi.org/10.1111/mec.13513][][10.1111/mec.13513]\from[url3]

{\bf TESS3: fast inference of spatial population structure and genome
scans for selection}\crlf
Kevin Caye, Timo M. Deist, Helena Martins, Olivier François\crlf
{\em Molecular ecology resources}, 2016, vol.~16, no 2, p.~540-548; doi:
\useURL[url4][http://dx.doi.org/10.1111/1755-0998.12471][][10.1111/1755-0998.12471]\from[url4]

\subsection[centres-dintérêt]{Centres d'intérêt}

\startdescription{Sports}
  Karaté Kyokushin en compétition, raid multipsort par équipe en
  compétition et sport de montagne (ski, randonnées, trekking, trail).
\stopdescription

\startdescription{Associatif}
  Membre du bureau des sports de l'ENSIMAG (2012-2013). Bénévole dans un
  projet de construction d'une bibliothèque à Caral au Pérou dans
  l'association étudiante solida'rire (2011-2012).
\stopdescription

\subsection[références]{Références}

\startitemize
\item
  \useURL[url5][http://membres-timc.imag.fr/Olivier.Francois/][][Olivier
  Francois]\from[url5], maître de conférence rattaché à l'ENSIMAG et au
  laboratoire TIMC-IMAG (La Tronche, France)
\item
  \useURL[url6][http://membres-timc.imag.fr/Michael.Blum/][][Michael
  Blum]\from[url6], directeur de recherche CNRS rattaché au laboratoire
  TIMC-IMAG et responsable de l'équipe BCM (La Tronche, France)
\item
  \useURL[url7][https://www.linkedin.com/in/everton-hermann-59908a6/][][Everton
  Hermann]\from[url7], ingénieur de recherche à Dassault Systèmes
  (Vélizy-Villacoublay, France)
\stopitemize

\thinrule

\startblockquote
\useURL[url8][mailto:kevin.caye@gmail.com][][kevin.caye@gmail.com]\from[url8]
• +33 6 19 45 11 46 • 26 ans\crlf
18 rue du pont prouiller - La Tronche 38700, France
\stopblockquote

\stoptext
